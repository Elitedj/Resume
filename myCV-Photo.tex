\documentclass[11pt]{article}


\setlength{\parindent}{0pt}
\usepackage{xltxtra}
\usepackage{hyperref}
\hypersetup{hidelinks}
\usepackage{url}
\urlstyle{tt}
\usepackage{xcolor}
\definecolor{CVBlue}{RGB}{23,110,191}
\usepackage{calc}
\usepackage{graphicx}
\usepackage{tikz}
\usetikzlibrary{calc}
\usepackage{fontspec}
\usepackage{xeCJK}
\CJKsetecglue{} %% 取消中文与数字之间间隙


%% Main document font
\setmainfont[
	Path = fonts/Main/,
	Extension = .otf ,
	BoldFont = texgyretermes-bold.otf, %加粗字体
]{texgyretermes-regular.otf} %正常字体

% 中文字体设置
\setCJKmainfont[
	Path = fonts/hansans/,
	Extension = .otf,
	BoldFont=SourceHanSansCN-Bold.otf, %加粗字体
]{SourceHanSansCN-Regular.otf} %正常字体


\usepackage{relsize}
\usepackage{xspace}

% use fontawesome
\usepackage{fontawesome} %一些图标

% a4纸,上下左右边距
\usepackage[
	a4paper,
	left=1.2cm,
	right=1.2cm,
	top=1.5cm,
	bottom=1cm,
	nohead
]{geometry}

\renewcommand{\baselinestretch}{1.5} %定义行间距1.5

\usepackage{titlesec}
\usepackage{enumitem}
\setlist{noitemsep} % removes spacing from items but leaves space around the whole list
%\setlist{nosep} % removes all vertical spacing within and around the list
\setlist[itemize]{topsep=0.25em, leftmargin=*}
\setlist[enumerate]{topsep=0.25em, leftmargin=*}



\titleformat{\section}         % Customise the \section command 
{\large\bfseries\raggedright} % Make the \section headers large (\Large),
% small capitals (\scshape) and left aligned (\raggedright)
{}{0em}                      % Can be used to give a prefix to all sections, like 'Section ...'
{}                           % Can be used to insert code before the heading
[{\color{CVBlue}\titlerule}]                 % Inserts a horizontal line after the heading
\titlespacing*{\section}{0cm}{*1.6}{*1.2}



\begin{document}
\pagenumbering{gobble}
	
%%%% 利用tikz来定位照片
\begin{tikzpicture}[remember picture, overlay] 
	\node[anchor = north east] at ($(current page.north east)+(-1cm,-1.2cm)$) {\includegraphics[height=2.5cm]{avatar}};
\end{tikzpicture}%

\centerline{\LARGE\bfseries{王大锤}}

\centerline{\normalsize{\faPhone\ 123-4567-8910 \quad \faEnvelopeO\ \href{mailto:123456789@qq.com}{123456789@qq.com}}}

\centerline{\normalsize{\faGithub\ https://github/Username \quad \faRssSquare \ http://yourBlogURL}}
	
\section{\makebox[\widthof{\faGraduationCap}][c]{\color{CVBlue}\faGraduationCap}\  教育背景}	
\textbf{麻省理工学院} \hfill 2000.9 -- \makebox[\widthof{2000.9}][s]{2004.7}

连网线 \quad 本科
\begin{itemize}
	\item 相关课程:吃、喝、玩、乐
\end{itemize}

\section{\makebox[\widthof{\faGraduationCap}][c]{\color{CVBlue}\faUsers}\ 项目经历}
\textbf{“流浪地球”计划} \hfill 2012.1

项目组长 \quad 负责行星发动机核心代码编写
\begin{itemize}
	\item 联合政府为使地球脱离太阳系,前往比邻星而开发的巨型机械设施,共有11000台,其中有10000台行星发动机安装在北半球的亚欧大陆和美洲大陆上。
	\item 另外赤道上还有2000台转向发动机。人类在北半球的亚欧大陆、美洲大陆和赤道上集中建造了无数庞大的行星发动机。行星发动机通过重聚变技术,来达到推动地球的目的。
	\item 北京第三交通局提醒您,道路千万条,安全第一条,行车不规范,亲人两行泪。
\end{itemize}
	
\section{\makebox[\widthof{\faGraduationCap}][c]{\color{CVBlue}\faList}\ 获奖情况}
\begin{itemize}
	\item 吃饭第一名 \hfill 2000.12
	
	\item 睡觉第一名 \hfill 2000.12

	\item 发呆第一名 \hfill 2000.12

	\item 喝水第一名 \hfill 2000.12

	\item 玩手机第一名 \hfill 2000.12

	\item 消费第一名 \hfill 2000.12
\end{itemize}

\section{\makebox[\widthof{\faGraduationCap}][c]{\color{CVBlue}\faCogs}\ IT 技能}

\begin{itemize}
	\item 精通各种BUG的编写
	\item 7天从入门到精通C++
	\item 熟练掌握各种品牌电脑的开机、关机、重启
	\item rm -rf
	\item 刷爆所有OJ题库
\end{itemize}
	
\section{\makebox[\widthof{\faGraduationCap}][c]{\color{CVBlue}\faInfo}\ 其他}

\begin{itemize}[parsep=0.5ex]
	\item 技术博客: http://yourBlogURL
	\item GitHub: https://github.com/Username
	\item 语言: 英语六级
\end{itemize}
	
%%%% 如果多页简历,可以手动在适当位置插入 \newpage 或者 \clearpage 开始新一页
	
\end{document}